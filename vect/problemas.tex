\documentclass[a4paper]{article}
\usepackage{amssymb,amsmath,bm,mathtools}
\usepackage[tmargin=2cm,lmargin=2.8cm]{geometry}
\usepackage[spanish]{babel}
\usepackage[T1]{fontenc}
\usepackage[utf8]{inputenc}
\newcommand{\ihat}{\boldsymbol{\hat{\imath}}}
\newcommand{\jhat}{\boldsymbol{\hat{\jmath}}}
\newcommand{\khat}{\boldsymbol{\hat{\bm{k}}}}
\newcommand{\euler}{\mathrm{e}}
\newcommand{\abs}[1]{\lVert #1 \rVert}
\newcommand{\problema}[1]{\textbf{\textit{Problema #1:}}}
\title{Problemas Selectos de Análisis Vectorial}
\author{S.N.L.}
\date{}
\begin{document}
\maketitle
\section*{Problemas del Primer Corte}
\problema{1.16}Si $\mathbf{\hat{a}}$ y $\mathbf{\hat{b}}$ son vectores unitarios y $\theta$ es el ángulo entre ellos, mostrar que:
\[\frac{1}{2}\abs{\mathbf{\hat{a}}-\mathbf{\hat{b}}}=\Big\lvert \sin\Big(\frac{\theta}{2}\Big)\Big\rvert\]
\problema{1.46}Sea $ABC$ un triángulo, $0$ cualquier punto de manera que $\vec{a}=\overrightarrow{0A},\vec{b}=\overrightarrow{0B},\vec{c}=\overrightarrow{0C}$.Mostrar que el área de $ABC$ es igual a:
\[S=\frac{1}{2}\abs{\vec{a}\times\vec{b}+\vec{b}\times\vec{c}+\vec{c}\times\vec{a}}\]
\problema{1.29}Obtener una expresión para $(\vec{A}\times\vec{B})\cdot(\vec{C}\times\vec{D})$
\end{document}