\documentclass[a4paper]{article}
\usepackage{amssymb,amsmath,bm,mathtools}
\usepackage[tmargin=2cm,lmargin=2.8cm]{geometry}
\usepackage[spanish]{babel}
\usepackage[T1]{fontenc}
\usepackage[utf8]{inputenc}
\newcommand{\ihat}{\boldsymbol{\hat{\imath}}}
\newcommand{\jhat}{\boldsymbol{\hat{\jmath}}}
\newcommand{\khat}{\boldsymbol{\hat{\bm{k}}}}
\newcommand{\euler}{\mathrm{e}}
\title{Notas de Análisis Vectorial}
\author{S.N.L.}
\date{}
\begin{document}
\maketitle
\section{Gráfica de una función y conjuntos de nivel}
Como sabemos, para $\mathbb{R}^3$ tenemos una base canónica:
\[\ihat=[1,0,0]\qquad \jhat=[0,1,0]\qquad \khat=[0,0,1]\]
De modo que cualquier vector en $\mathbb{R}^3$ puede ser expresado como:
\[\vec{r}=x\ihat+y\jhat+z\khat \]
o de forma equivalente:
\[\vec{r}=[x,y,z]\]
Dando una generalización:
\[\wp_1=[1,0,0,\ldots,0]\]
\[\wp_2=[0,1,0,\ldots,0]\]
\[\vdots\]
\[\wp_n=[0,0,0,\ldots,n]\]
es la base canónica de $\mathbb{R}^n$ de modo que cualquier vector $\vec{x}$ de $\mathbb{R}^n$ se puede expresar como:
\[\vec{x}=x_1\wp_1+x_2\wp_2+x_3\wp_3+\ldots\,+x_n\wp_n\]
o de forma equivalente
\[\vec{x}=[x_1,x_2,x_3,\ldots,x_n]\]
Se dice que una función $f(x)$ es una función escalar de varias variables si: $f: U \subset \mathbb{R}^n \rightarrow \mathbb{R}$\\Por ejemplo:\\
\hspace*{3em} Sea $f:U\subset\mathbb{R}^4\rightarrow\mathbb{R}$ con $[x_1,x_2,x_3,x_4]\longrightarrow \frac{x_1+x_2}{x_3+x_4}$\\
Una función vectorial de variables reales es una función de la forma:
\[\vec{f}=U\subset\mathbb{R}^n\longrightarrow\mathbb{R}^m\]
\begin{center}
\small{(la función en $\mathbb{R}^n$ se puede encontrar en $\mathbb{R}^m$)}\\
\end{center}
Si $\beta_1=\{\wp_1,\wp_2,\ldots,\wp_n\}$ y $\beta_2=\{\wp_1,\wp_2,\ldots,\wp_m\}$
son las bases canónicas de $\mathbb{R}_n$ y de $\mathbb{R}_m$ respectivamente, luego:
\[[x_1,x_2,x_3,\ldots,x_n]=[f_1,f_2,f_3,\ldots,f_m]\]
Donde
\[f_1=f_1(x^{-1}=f(x_1,x_2,x_3,\ldots,x_n)\]
son funciones escalares, luego:
\[\vec{f}(x_1,x_2,x_3,\ldots,x_n)=[f_1,f_2,f_3,\ldots,f_m]\]
\subsection{Ejemplos}
Ejemplo 1:
\[
\begin{split}
&\vec{f}:\mathbb{R}^3\longrightarrow\mathbb{R}^4\\
&[x,y,z]\longmapsto[xy-xz,x^2y+x^2z^2,xyz,xy-yz]
\end{split}
\]

Ejemplo 2:
\[
\begin{split}
&\vec{f}:\mathbb{R}^4\longrightarrow\mathbb{R}^3\\
&[x_1,x_2,x_3,x_4]\longmapsto[x_1^2-x_2^2,x_3^2-x_4^2,x_4^2-x_2^2]
\end{split}
\]
\subsection{Gráficas de funciones y conjuntos de nivel}
Si $f:U\subset\mathbb{R}^n\rightarrow\mathbb{R}$ entonces la gráfica de $f$ se define como:
\[grafica\ de\ f=\{\vec{x}\in\mathbb{R}^{n+1}\big|\vec{x}=[x_1,x_2,\ldots,x_n,f(x_1,x_2,\ldots,x_n)]\}\]
Si $f:U\subset\mathbb{R}^n\longrightarrow\mathbb{R}$, luego el conjunto de nivel de $f$ se define como:
\[conjunto\ de\ nivel\ de\ f:\{\vec{x}\in U\big|f(\vec{x})=C\}\]
{\small Nota: El conjunto de nivel siempre esta contenido en el dominio de $f$}
\section{Límites y Continuidad de Funciones Vectoriales}
Una función vectorial del tipo $\vec{f}:\mathbb{R}\longrightarrow\mathbb{R}^2$ con $\vec{f}(t)=[f_1(t),f_2(t),f_3(t)]$
o bien $\vec{f}(t)=f_1\ihat+f_2\jhat+f_3\khat$ representa una curva en $\mathbb{R}^3$\\
Si $p(x,y,z)$ es cualquier punto de la curva $C$, entonces $x=f_1(t)$, $y=f_2(t)$, $y=f_3(t)$

Ejercicio 3.8: Mostrar que si $\vec{f}(t)$ es continua en $t=t_0$, es decir:
\[\lim_{t\rightarrow t_0} \vec{f}(t)\]
entonces:
\[\lim_{t\rightarrow t_0}\vec{f}_i(t)=\vec{f}_i(t_0)\]
Solución: Como
\[\lim_{t\rightarrow t_0}\vec{f}_i(t)=\vec{f}_i(t_0)\]
es decir:
\begin{equation}\lim_{t\rightarrow t_0} \vec{f}(t)=[f_1(t_0),f_2(t_0),f_3(t_0)] \end{equation}
Por otro lado:
\begin{equation}
\lim_{t\rightarrow t_0} \vec{f}(t)=\Big[\lim_{t\rightarrow t_0} f_1(t),\lim_{t\rightarrow t_0} f_2(t),\lim_{t\rightarrow t_0} f_3(t)\Big]
\end{equation}
De 1 y 2 se tiene por igualdad  de vectores que:
\[\lim_{t\rightarrow t_0} f_1(t)=f_1(t_0),\lim_{t\rightarrow t_0} f_2(t)=f_2(t_0),\lim_{t\rightarrow t_0} f_3(t)=f_3(t_0)\]
Es decir, las funciones escalares son continuas.
\section{Derivada de una función vectorial}
Si $\vec{f}$ es una función vectorial de la forma $\vec{f}:\mathbb{R}\longrightarrow\mathbb{R}^3$ en $\vec{f}(t)=[f_1(t),f_2(t),f_3(t)]$ entonces la derivada de $\vec{f}$ con respecto de $t$ (si existe) que se denota como:
\[\frac{d\vec{f}(t)}{dt}\qquad o\qquad f'(t)\]
y se define como:
\[\vec{f}'(t)=\lim_{\Delta_t\rightarrow 0} \frac{\vec{f}(t+\Delta_t - \vec{f}(t)}{\Delta_t}\]
\subsection{Propiedades de las derivadas}
Si $\phi(t),\vec{A}(t),\vec{B}(t)\ y\ \vec{C}(t)$ son funciones escalares y vectoriales respectivamente, entonces se cumple:
\begin{enumerate}
{\item $\frac{d}{dt}[\phi(t)\vec{A}(t)]=\phi'(t)\vec{A}(t)+\phi(t)\vec{A}'(t)$}
{\item $\frac{d}{dt}[\vec{A}(t)\cdot\vec{B}(t)]=\vec{A}'(t)\cdot\vec{B}(t)+\vec{A}(t)\cdot\vec{B}'(t)]$}
{\item $\frac{d}{dt}[\vec{A}(t)\times\vec{B}(t)]=\vec{A}'(t)\times\vec{B}(t)+\vec{A}(t)\times\vec{B}'(t)$}
{\item $\frac{d}{dt}[\vec{A}(t)\cdot\vec{B}(t)\times\vec{C}(t)]=\vec{A}'(t)\cdot\vec{B}(t)\times\vec{C}(t)+\vec{A}(t)\cdot\vec{B}'(t)\times\vec{C}(t)+\vec{A}(t)\cdot\vec{B}(t)\times\vec{C}'(t)$}
\end{enumerate}
\subsection{Un poco de ecuaciones diferenciales}
Una ecuación diferencial ordinaria de orden $n$ se puede expresar como:
\[f(x,y,y',y'',\ldots,y^{(n-1)})=0\]
Una ecuación diferencial ordinaria lineal homogénea con coeficientes constantes de orden $n$ se representa como:
\begin{equation}
a_ny^n+a_{n-1}y^{n-1}+\ldots+a_1y'+a_0y=0
\end{equation}
Para resolver 3 se propone:
\[y=c\euler\]
Luego:
\begin{align*}
y'&=cp\euler^{px}\\
y''&=cp^2\euler^{px}\\
\vdots&\hspace{2em}\vdots\\
y^{(n)}&=cp^n\euler^{px}
\end{align*}
Por lo tanto:
\begin{gather*}
a_ncp^n\euler^{px}+a_{n-1}p^{n-1}\euler^{px}+\ldots+a_1cp\euler^{px}+a_0\euler^{px}=0\\
(a_np^n+a_{n-1}p^{n-1}+\ldots +a_1p+a_0)c\euler^{px}=0
\end{gather*}
De donde:
\begin{equation}
\label{eq4}
a_np^n+a_{n-1}p^{n-1}+\ldots+a_1p+a_0=0
\end{equation}
Por el Teorema Fundamental del Álgebra, el polinomio \ref{eq4} tiene a lo mas $n$ raices reales y complejas.\\
Seab $P_1,P_2,\ldots,P_n$ las raíces de 4, la solución general es:
\[y=c_1\euler^{px}+c_2\euler^{p_2x}+\ldots+c_n\euler^{p_{n}x}\]
\subsection{Derivadas Parciales}
Sea $f:U\subset\mathbb{R}^n\longrightarrow\mathbb{R}$ una función escalar, así $f=f(x_1,x_2,\ldots,x_n)$. Si ${\wp_1,\wp_2,\ldots,\wp_n}$ es la base canónica de $\mathbb{R}^n$, entonces:
\[\vec{x}=x_1\wp_1+x_2\wp_2+\ldots+x_n\wp_n\]
Luego, \textbf{la derivada parcial de $f$ con respecto a la variable $x_j$}, que se denota como $\frac{\delta f}{\delta x_j}$, se define como:
\[\lim_{h\rightarrow 0} \frac{f(x_1,x_2,\ldots,x_n+h)-f(x_1,x_2,\ldots,x_n)}{h}\]
O bien:
\[\frac{\delta f}{\delta x_j}=\lim_{h\rightarrow 0}\frac{f(\vec{x}+h\wp_j)-f(\vec{x}}{h})\]
Si $f$ tiene mas de una derivada, estas se denotan como:
\[\frac{\delta^2 f}{\delta x_1^2}\ ,\frac{\delta^2 f}{\delta x_1 \delta x_2}\]
O también, si $f=f(u,v,w)$:
\[\frac{\delta^2 f}{\delta u \delta v}=f_{vu}\]
\subsection{Propiedades de las Derivadas Parciales}
Si $\vec{A}=\vec{A}(u,v,w) y\vec{B}=\vec{B}(u,v,w) y \phi=\phi(u,v,w)$ son funciones vectoriales y escalares diferenciables, entonces se cumple:
\begin{enumerate}
{\item $\frac{\delta}{\delta u}(\phi\vec{A})=\phi_u\vec{A}+\phi\vec{A}_u$}
{\item $\frac{\delta}{\delta u}(\vec{A}+\vec{B})=\vec{A}_u+\vec{B}_u$}
{\item $\frac{\delta}{\delta u}(\vec{A}\cdot\vec{B})=\vec{A}_b\cdot\vec{B}+\vec{A}\cdot\vec{B}_u$}
{\item $\frac{\delta}{\delta u}(\vec{A}\times\vec{B})=\vec{A}_u\times\vec{B}+\vec{A}\times\vec{B}_u$}
\end{enumerate}
\subsection{Longitud de Arco}
Si una curva $C$ en $\mathbb{R}^3$ esta representada por:
\begin{equation}
\begin{cases}
x&=f_1(t)\\
y&=f_2(t)\\
z&=f_3(t)
\end{cases}
\end{equation}
Del cálculo básico, el elemento diferencial de longitud de arco es:
\[ds^2=dx^2+dy^2+dz^2\]
Dividiendo entre $dt^2$ se tiene:
\[\frac{ds^2}{dt^2}=\frac{dx^2}{dt^2}+\frac{dy^2}{dt^2}+\frac{dz^2}{dt^2}\]
Por las ecuaciones 5 se tiene:
\[\Big(\frac{ds}{dt}\Big)^2[f'_1(t)]^2+[f'_2(t)]^2+[f'_3(t)]^2\]
Luego:
\[\Big(\frac{ds}{dt}\Big)^2=\lVert\vec{f'}(t)\rVert\]
i.e.
\[\frac{ds}{dt}=\lVert\vec{f'}(t)\rVert\]
Luego:
\[ds=\lVert\vec{f'}(t)\rVert dt\]
Si la curva $C$ esta definida en un intervalo $a\leq t\leq b$, luego:
\begin{equation}
\label{eqarco}
S=\int_a^b \lVert\vec{f'}(t)\rVert dt
\end{equation}
Si se tiene una curva plana $C$ en el plano $xy$ representada por $y=y(x)$ con $a\leq x \leq b$ luego:
\[\vec{r}=x\ihat+y(x)\jhat\] 
luego:
\[\vec{r'}(x)=\ihat+y'(x)\jhat\]
así:
\[\lVert \vec{r'}(x)\rVert=\sqrt{1+[y'(x)]^2}\]
De done la longitud de arco $C$ de $a$ a $b$ es:
\[S=\int_a^b\sqrt{1+[y'(x)]^2} dx\]
Si en la ecuación \ref{eqarco} se deja libre el límite superior, entonces se tiene que es una función de una variable:
\begin{equation}
S(t)=\int_0^t \lVert \vec{f'}(\mathfrak{z})\lVert d\mathfrak{z}
\end{equation}
como:
\[\frac{ds(t)}{dt}=\lVert f'(t)\rVert\]
Luego, $s(t)$ es una función estrictamente creciente, de modo que admite una única inversa 
denominada $t=q(s)$.Así la curva $C$ representada por:
\begin{equation}
\vec{r}(t)=x(t)\ihat+y(t)\jhat+z(t)\khat \qquad a\leq t\leq b
\end{equation}
y se puede representar como:
\[\vec{r}[q(s)]=x[q(s)]\ihat+y[q(s)]\jhat+z[q(s)]\khat\qquad 0\leq s\leq l\]
o bien:
\begin{equation}
\vec{r}(s)=x(s)\ihat+y(s)\jhat+k(s)\khat\qquad 0\leq s\leq l
\end{equation}
\textbf{Ejercicio}:Obtener un vector unitario tangente al circulo de radio $a$ en $t=\pi/2$\\[2ex]
\textbf{Solución}:Sabemos que $f(t)=a\cos(t)\ihat+a\sen(t)\jhat$ y $f''(t)=-a\cos(t)\ihat+a\cos(t)\jhat$, dado que:
\[\lVert f''(t)\rVert=a\]
luego:
\[\mathbf{\hat{T}}=\frac{f''(t)}{\lVert f''(t) \rVert}=-\sin(t)\ihat+\cos(t)\jhat\]
para $t=\pi/2$ se tiene:
\[\mathbf{\hat{T}}(\pi/2)=-\ihat\]
\subsection{Aplicaciones a la Geometría Diferencial}
Sea $C$ una curva en el espacio $\mathbb{R}^3$ definida por:
\[\vec{r}(t)=x(t)\ihat+y(t)\jhat+z(t)\khat \qquad a\leq t\leq b\]
O bien:
\[\vec{r}(s)=x(s)\ihat+y(s)\jhat+z(s)\khat \qquad 0\leq s\leq l\]
El vector \textbf{normal principal} a la curva $C$ en cada punto se define como:
\begin{equation}
\label{vecnorm}
\frac{dt\mathbf{\hat{T}}}{ds}=\mathbf{K}\mathbf{\hat{N}}
\end{equation}
Donde $\mathbf{K}$ es la \textbf{curvatura} de $C$ y $\rho=\frac{1}{\mathrm{K}}$ es el radio de curvatura de $C$ y es:
\[\mathbf{K}=\Big\lVert \frac{d\mathbf{\hat{T}}}{ds}\Big\rVert\]
Se sabe que $\frac{d\mathbf{\hat{T}}}{ds}\perp\mathbf{\hat{T}}$, como
\[\frac{d\mathbf{\hat{T}}}{ds}\parallel\mathbf{\hat{N}}\]
se tiene que:
\[\mathbf{\hat{T}}\perp\mathbf{\hat{N}}\]
i.e. $\mathbf{\hat{T}}\cdot\mathbf{\hat{N}}=0$
El vector \textbf{binomial} $\mathbf{\hat{B}}$ se define como:
\begin{equation}
\label{vecbinomial}
\mathbf{\hat{B}}=\mathbf{\hat{T}}\times\mathbf{\hat{N}}
\end{equation}
Asi como se denota ,$\mathbf{\hat{B}}$ es un vector unitario, pues:
\[\lVert \mathbf{\hat{B}} \rVert=\lVert \mathbf{\hat{T}} \rVert\lVert \mathbf{\hat{N}} \rVert\sin(90)\]
La torsión de la curva $C$ se define como:
\begin{equation}
 \frac{d\mathbf{\hat{B}}}{ds}=-\vartheta\mathbf{\hat{N}}
\end{equation}
Por lo anterior se tiene que los vectores $\mathbf{\hat{T}},\mathbf{\hat{N}},\mathbf{\hat{B}}$ forman una terna positiva o un sistema derecho que de \ref{vecbinomial} se tiene:
\begin{equation}
\label{vectall}
\mathbf{\hat{B}}=\mathbf{\hat{T}}\times\mathbf{\hat{N}}\qquad\mathbf{\hat{N}}=\mathbf{\hat{B}}\times\mathbf{\hat{T}}\qquad\mathbf{\hat{T}}=\mathbf{\hat{N}}\times\mathbf{\hat{B}}
\end{equation}
\textbf{Ejercicio}:Obtener una expresión para $\frac{d\mathbf{\hat{T}}}{ds}$\\[2ex]
\textbf{Solución:}:Por las ecuaciones \ref{vectall}:
\[\mathbf{\hat{N}}=\mathbf{\hat{B}}\times\mathbf{\hat{T}}\]
luego:
\[\frac{d\mathbf{\hat{N}}}{ds}=\frac{d\mathbf{\hat{B}}}{ds}\times\mathbf{\hat{T}}+\mathbf{\hat{B}}\times\frac{d\mathbf{\hat{N}}}{ds}\]
Por las ecuaciones \ref{vecbinomial} y \ref{vecnorm}:
\begin{align*}
\frac{d\mathbf{\hat{N}}}{ds}&=-\vartheta\mathbf{\hat{N}}\times\mathbf{\hat{B}}\times(\mathbf{K}\mathbf{\hat{N}})\\
\frac{d\mathbf{\hat{N}}}{ds}&=-\vartheta\mathbf{\hat{N}}\times\mathbf{\hat{T}}+\mathbf{K}\mathbf{\hat{B}}\times\mathbf{\hat{N}}
\end{align*}
por las ecuaciones \ref{vectall}:
\begin{equation}
\label{vectdn}
\frac{d\mathbf{\hat{N}}}{ds}=\vartheta\mathbf{\hat{B}}-\mathbf{K}\mathbf{\hat{T}}
\end{equation}
Las ecuaciones \ref{vecnorm}, \ref{vecbinomial},\ref{vectdn} son llamadas las \textbf{ecuaciones de Frenet-Serret}:
\begin{align*}
\frac{d\mathbf{\hat{T}}}{ds}&=\mathbf{K}\mathbf{\hat{N}}\\
\frac{d\mathbf{\hat{B}}}{ds}&=-\vartheta\mathbf{\hat{N}}\\
\frac{d\mathbf{\hat{N}}}{ds}&=\vartheta\mathbf{\hat{B}}-\mathbf{K}\mathbf{\hat{T}}
\end{align*}\\
Si $C$ es la curva representada por $\vec{r}(t)=x(t)\ihat+y(t)\jhat+z(t)\khat\ ,a\leq t\leq b$, entonces:
\begin{align*}
\mathbf{K}&=\frac{\lVert \vec{r'}(t)\times\vec{r''}\rVert}{\lVert \vec{r'}(t)\rVert^3}\\
\vartheta&=\frac{[\vec{r'}(t)\quad\vec{r''}(t)\quad\vec{r'''}(t)]}{\lVert \vec{r'}(t)\times\vec{r''}(t)\rVert^2}
\end{align*}
También:
\begin{align*}
\mathbf{\hat{T}}&=\frac{\vec{r'}(t)}{\lVert \vec{r'}(t)}\\
\mathbf{\hat{B}}&=\frac{\vec{r'}(t)\times\vec{r''}(t)}{\rVert \vec{r'}(t)\times\vec{r''}(t)\lVert}\\
\mathbf{\hat{N}}&=\frac{[\vec{r'}(t)\times\vec{r''}(t)]\times\vec{r'}(t)}{\lVert [\vec{r'}(t)\times\vec{r''}(t)]\times\vec{r'}(t)\rVert}
\end{align*}
\section{Coordenadas curvilíneas}
\subsection{Coordenadas polares}
Si $P(x,y)$ es un punto del plano, de manera que:
\begin{gather*}
x=r\cos \phi\\
y=r\sin \phi
\end{gather*}
y también se tiene:
\begin{gather*}
r=\sqrt{x^2+y^2}\quad r\in[0,\infty]\\
\phi=\tan^{-1}\Big(\frac{y}{x}\Big)\quad\phi\in[0,2\pi] 
\end{gather*}
luego:
\begin{gather}
\tag{Sistema cartesiano}
\vec{r}=x\ihat+y\jhat\\
\tag{Sistema polar}
\vec{r}=r\cos\ \phi\ihat+r\sin\ \phi\jhat
\end{gather}
\subsection{Coordenadas cilíndricas}
Si $P(x,y,z)$ es cualquier punto en $\mathbb{R}^3$, entonces:
\begin{gather*}
x=\rho\cos\ \phi\\
y=\rho\sin\ \phi\\
z=z
\end{gather*}
y también:
\begin{gather*}
\rho=\sqrt{x^2+y^2}\quad\rho\geq 0\\
\phi=\tan^{-1}\Big(\frac{y}{x}\Big)\quad\phi\in[0,2\pi] 
z=z
\end{gather*}
luego:
\begin{gather}
\tag{Sistema cartesiano}
\vec{r}=x\ihat+y\jhat+z\khat\\
\tag{Sistema cilindrico}
\vec{r}=\rho\cos\ \phi\ihat+\rho\sin\ \phi\jhat+z\khat
\end{gather}
\subsection{Coordenadas esféricas}
Si $P(x,y,z)$ es cualquier punto en $\mathbb{R}^3$, entonces:
\begin{gather*}
x=r\sin\theta\cos\phi\\
y=r\sin\theta\sin\phi\\
z=r\cos\theta 
\end{gather*}
luego:
\begin{gather*}
r=\sqrt{x^2+y^2+z^2}\quad r\geq 0\\
\theta=\cos^{-1}\Bigg(\frac{z}{\sqrt{x^2+y^2+z^2}}\Bigg)\quad 0\leq\theta\leq\pi\\
\phi=\tan^{-1}\Big(\frac{y}{x}\Big)\quad\phi\in[0,2\pi] 
\end{gather*}
de modo que:
\begin{gather}
\tag{Sistema cartesiano}
\vec{r}=x\ihat+y\jhat+z\khat\\
\tag{Sistema esférico}
\vec{r}=r\sin\theta\cos\phi\ihat+r\sin\theta\sin\phi\jhat+r\cos\theta\khat 
\end{gather}
\end{document}