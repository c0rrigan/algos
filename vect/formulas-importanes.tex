\documentclass[a4paper]{article}
\usepackage{amssymb,amsmath,bm,mathtools}
\usepackage[tmargin=2cm,lmargin=2.8cm]{geometry}
\usepackage[spanish]{babel}
\usepackage[T1]{fontenc}
\usepackage[utf8]{inputenc}
\newcommand{\ihat}{\boldsymbol{\hat{\imath}}}
\newcommand{\jhat}{\boldsymbol{\hat{\jmath}}}
\newcommand{\khat}{\boldsymbol{\hat{\bm{k}}}}
\newcommand{\euler}{\mathrm{e}}
\title{Formulas Importantes}
\author{S.N.L.}
\begin{document}
\maketitle
\section{Calculo Diferencial Vectorial}
\begin{align*}
\mathbf{\hat{T}}&=\frac{\vec{r'}(t)}{\lVert \vec{r'}(t)\rVert}\\
\mathbf{\hat{n}}&=\frac{\vec{r}_u\times\vec{r}_v}{\rVert\vec{r}_u\times\vec{r}_v\rVert}
\end{align*}
\section{Geometría Diferencial}
Si $C$ es la curva representada por $\vec{r}(t)=x(t)\ihat+y(t)\jhat+z(t)\khat\ ,a\leq t\leq b$, entonces:
\begin{align*}
\mathbf{K}&=\frac{\lVert \vec{r'}(t)\times\vec{r''}\rVert}{\lVert \vec{r'}(t)\rVert^3}\\
\vartheta&=\frac{[\vec{r'}(t)\quad\vec{r''}(t)\quad\vec{r'''}(t)]}{\lVert \vec{r'}(t)\times\vec{r''}(t)\rVert^2}
\end{align*}
También:
\begin{align*}
\mathbf{\hat{T}}&=\frac{\vec{r'}(t)}{\lVert \vec{r'}(t)\rVert}\\
\mathbf{\hat{B}}&=\frac{\vec{r'}(t)\times\vec{r''}(t)}{\rVert \vec{r'}(t)\times\vec{r''}(t)\lVert}\\
\mathbf{\hat{N}}&=\frac{[\vec{r'}(t)\times\vec{r''}(t)]\times\vec{r'}(t)}{\lVert [\vec{r'}(t)\times\vec{r''}(t)]\times\vec{r'}(t)\rVert}
\end{align*}
\subsection{Coordenadas cilíndricas}
Si $P(x,y,z)$ es cualquier punto en $\mathbb{R}^3$, entonces:
\begin{gather*}
x=\rho\cos\ \phi\\
y=\rho\sin\ \phi\\
z=z
\end{gather*}
y también:
\begin{gather*}
\rho=\sqrt{x^2+y^2}\quad\rho\geq 0\\
\phi=\tan^{-1}\Big(\frac{y}{x}\Big)\quad\phi\in[0,2\pi] 
z=z
\end{gather*}
luego:
\begin{gather}
\tag{Sistema cartesiano}
\vec{r}=x\ihat+y\jhat+z\khat\\
\tag{Sistema cilindrico}
\vec{r}=\rho\cos\ \phi\ihat+\rho\sin\ \phi\jhat+z\khat
\end{gather}
\subsection{Coordenadas esféricas}
Si $P(x,y,z)$ es cualquier punto en $\mathbb{R}^3$, entonces:
\begin{gather*}
x=r\sin\theta\cos\phi\\
y=r\sin\theta\sin\phi\\
z=r\cos\theta 
\end{gather*}
luego:
\begin{gather*}
r=\sqrt{x^2+y^2+z^2}\quad r\geq 0\\
\theta=\cos^{-1}\Bigg(\frac{z}{\sqrt{x^2+y^2+z^2}}\Bigg)\quad 0\leq\theta\leq\pi\\
\phi=\tan^{-1}\Big(\frac{y}{x}\Big)\quad\phi\in[0,2\pi] 
\end{gather*}
de modo que:
\begin{gather}
\tag{Sistema cartesiano}
\vec{r}=x\ihat+y\jhat+z\khat\\
\tag{Sistema esférico}
\vec{r}=r\sin\theta\cos\phi\ihat+r\sin\theta\sin\phi\jhat+r\cos\theta\khat 
\end{gather}
\end{document}
